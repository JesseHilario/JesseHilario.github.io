% Options for packages loaded elsewhere
\PassOptionsToPackage{unicode}{hyperref}
\PassOptionsToPackage{hyphens}{url}
%
\documentclass[
]{article}
\usepackage{amsmath,amssymb}
\usepackage{lmodern}
\usepackage{iftex}
\ifPDFTeX
  \usepackage[T1]{fontenc}
  \usepackage[utf8]{inputenc}
  \usepackage{textcomp} % provide euro and other symbols
\else % if luatex or xetex
  \usepackage{unicode-math}
  \defaultfontfeatures{Scale=MatchLowercase}
  \defaultfontfeatures[\rmfamily]{Ligatures=TeX,Scale=1}
\fi
% Use upquote if available, for straight quotes in verbatim environments
\IfFileExists{upquote.sty}{\usepackage{upquote}}{}
\IfFileExists{microtype.sty}{% use microtype if available
  \usepackage[]{microtype}
  \UseMicrotypeSet[protrusion]{basicmath} % disable protrusion for tt fonts
}{}
\makeatletter
\@ifundefined{KOMAClassName}{% if non-KOMA class
  \IfFileExists{parskip.sty}{%
    \usepackage{parskip}
  }{% else
    \setlength{\parindent}{0pt}
    \setlength{\parskip}{6pt plus 2pt minus 1pt}}
}{% if KOMA class
  \KOMAoptions{parskip=half}}
\makeatother
\usepackage{xcolor}
\usepackage[margin=1in]{geometry}
\usepackage{color}
\usepackage{fancyvrb}
\newcommand{\VerbBar}{|}
\newcommand{\VERB}{\Verb[commandchars=\\\{\}]}
\DefineVerbatimEnvironment{Highlighting}{Verbatim}{commandchars=\\\{\}}
% Add ',fontsize=\small' for more characters per line
\usepackage{framed}
\definecolor{shadecolor}{RGB}{248,248,248}
\newenvironment{Shaded}{\begin{snugshade}}{\end{snugshade}}
\newcommand{\AlertTok}[1]{\textcolor[rgb]{0.94,0.16,0.16}{#1}}
\newcommand{\AnnotationTok}[1]{\textcolor[rgb]{0.56,0.35,0.01}{\textbf{\textit{#1}}}}
\newcommand{\AttributeTok}[1]{\textcolor[rgb]{0.77,0.63,0.00}{#1}}
\newcommand{\BaseNTok}[1]{\textcolor[rgb]{0.00,0.00,0.81}{#1}}
\newcommand{\BuiltInTok}[1]{#1}
\newcommand{\CharTok}[1]{\textcolor[rgb]{0.31,0.60,0.02}{#1}}
\newcommand{\CommentTok}[1]{\textcolor[rgb]{0.56,0.35,0.01}{\textit{#1}}}
\newcommand{\CommentVarTok}[1]{\textcolor[rgb]{0.56,0.35,0.01}{\textbf{\textit{#1}}}}
\newcommand{\ConstantTok}[1]{\textcolor[rgb]{0.00,0.00,0.00}{#1}}
\newcommand{\ControlFlowTok}[1]{\textcolor[rgb]{0.13,0.29,0.53}{\textbf{#1}}}
\newcommand{\DataTypeTok}[1]{\textcolor[rgb]{0.13,0.29,0.53}{#1}}
\newcommand{\DecValTok}[1]{\textcolor[rgb]{0.00,0.00,0.81}{#1}}
\newcommand{\DocumentationTok}[1]{\textcolor[rgb]{0.56,0.35,0.01}{\textbf{\textit{#1}}}}
\newcommand{\ErrorTok}[1]{\textcolor[rgb]{0.64,0.00,0.00}{\textbf{#1}}}
\newcommand{\ExtensionTok}[1]{#1}
\newcommand{\FloatTok}[1]{\textcolor[rgb]{0.00,0.00,0.81}{#1}}
\newcommand{\FunctionTok}[1]{\textcolor[rgb]{0.00,0.00,0.00}{#1}}
\newcommand{\ImportTok}[1]{#1}
\newcommand{\InformationTok}[1]{\textcolor[rgb]{0.56,0.35,0.01}{\textbf{\textit{#1}}}}
\newcommand{\KeywordTok}[1]{\textcolor[rgb]{0.13,0.29,0.53}{\textbf{#1}}}
\newcommand{\NormalTok}[1]{#1}
\newcommand{\OperatorTok}[1]{\textcolor[rgb]{0.81,0.36,0.00}{\textbf{#1}}}
\newcommand{\OtherTok}[1]{\textcolor[rgb]{0.56,0.35,0.01}{#1}}
\newcommand{\PreprocessorTok}[1]{\textcolor[rgb]{0.56,0.35,0.01}{\textit{#1}}}
\newcommand{\RegionMarkerTok}[1]{#1}
\newcommand{\SpecialCharTok}[1]{\textcolor[rgb]{0.00,0.00,0.00}{#1}}
\newcommand{\SpecialStringTok}[1]{\textcolor[rgb]{0.31,0.60,0.02}{#1}}
\newcommand{\StringTok}[1]{\textcolor[rgb]{0.31,0.60,0.02}{#1}}
\newcommand{\VariableTok}[1]{\textcolor[rgb]{0.00,0.00,0.00}{#1}}
\newcommand{\VerbatimStringTok}[1]{\textcolor[rgb]{0.31,0.60,0.02}{#1}}
\newcommand{\WarningTok}[1]{\textcolor[rgb]{0.56,0.35,0.01}{\textbf{\textit{#1}}}}
\usepackage{graphicx}
\makeatletter
\def\maxwidth{\ifdim\Gin@nat@width>\linewidth\linewidth\else\Gin@nat@width\fi}
\def\maxheight{\ifdim\Gin@nat@height>\textheight\textheight\else\Gin@nat@height\fi}
\makeatother
% Scale images if necessary, so that they will not overflow the page
% margins by default, and it is still possible to overwrite the defaults
% using explicit options in \includegraphics[width, height, ...]{}
\setkeys{Gin}{width=\maxwidth,height=\maxheight,keepaspectratio}
% Set default figure placement to htbp
\makeatletter
\def\fps@figure{htbp}
\makeatother
\setlength{\emergencystretch}{3em} % prevent overfull lines
\providecommand{\tightlist}{%
  \setlength{\itemsep}{0pt}\setlength{\parskip}{0pt}}
\setcounter{secnumdepth}{-\maxdimen} % remove section numbering
\ifLuaTeX
  \usepackage{selnolig}  % disable illegal ligatures
\fi
\IfFileExists{bookmark.sty}{\usepackage{bookmark}}{\usepackage{hyperref}}
\IfFileExists{xurl.sty}{\usepackage{xurl}}{} % add URL line breaks if available
\urlstyle{same} % disable monospaced font for URLs
\hypersetup{
  pdftitle={Case Study BellaBeat},
  hidelinks,
  pdfcreator={LaTeX via pandoc}}

\title{Case Study BellaBeat}
\author{}
\date{\vspace{-2.5em}2022-07-27}

\begin{document}
\maketitle

\hypertarget{scenario}{%
\subsection{Scenario}\label{scenario}}

You are a junior data analyst working on the marketing analyst team at
Bellabeat, a high-tech manufacturer of health-focused products for
women. Bellabeat is a successful small company, but they have the
potential to become a larger player in the global smart device market.
Urška Sršen, cofounder and Chief Creative Officer of Bellabeat, believes
that analyzing smart device fitness data could help unlock new growth
opportunities for the company. You have been asked to focus on one of
Bellabeat's products and analyze smart device data to gain insight into
how consumers are using their smart devices. The insights you discover
will then help guide marketing strategy for the company. You will
present your analysis to the Bellabeat executive team along with your
high-level recommendations for Bellabeat's marketing strategy.

\hypertarget{ask}{%
\subsection{Ask}\label{ask}}

\hypertarget{business-task}{%
\paragraph{Business Task}\label{business-task}}

Analyze non-Bellabeat smart device usage data in order to gain insight
into how the the Leaf tracker, Bellabeat's classic wellness tracker,
might be used by consumers. Our insights can drive business decisions by
adapting marketing strategy to target the consumers who use
non-Bellabeat smart devices and how they specifically use their devices.

\hypertarget{guiding-questions}{%
\paragraph{Guiding Questions}\label{guiding-questions}}

\begin{enumerate}
\def\labelenumi{\arabic{enumi}.}
\tightlist
\item
  What are some trends in smart device usage?
\item
  How could these trends apply to Bellabeat customers?
\item
  How could these trends help influence Bellabeat marketing strategy?
\end{enumerate}

\hypertarget{prepare}{%
\subsection{Prepare}\label{prepare}}

\hypertarget{loading-packages}{%
\paragraph{Loading Packages}\label{loading-packages}}

\begin{Shaded}
\begin{Highlighting}[]
\FunctionTok{library}\NormalTok{(tidyverse)}
\end{Highlighting}
\end{Shaded}

\begin{verbatim}
## -- Attaching packages --------------------------------------- tidyverse 1.3.2 --
## v ggplot2 3.3.6     v purrr   0.3.4
## v tibble  3.1.7     v dplyr   1.0.9
## v tidyr   1.2.0     v stringr 1.4.0
## v readr   2.1.2     v forcats 0.5.1
## -- Conflicts ------------------------------------------ tidyverse_conflicts() --
## x dplyr::filter() masks stats::filter()
## x dplyr::lag()    masks stats::lag()
\end{verbatim}

\begin{Shaded}
\begin{Highlighting}[]
\FunctionTok{library}\NormalTok{(janitor)}
\end{Highlighting}
\end{Shaded}

\begin{verbatim}
## 
## Attaching package: 'janitor'
## 
## The following objects are masked from 'package:stats':
## 
##     chisq.test, fisher.test
\end{verbatim}

\begin{Shaded}
\begin{Highlighting}[]
\FunctionTok{library}\NormalTok{(lubridate)}
\end{Highlighting}
\end{Shaded}

\begin{verbatim}
## 
## Attaching package: 'lubridate'
## 
## The following objects are masked from 'package:base':
## 
##     date, intersect, setdiff, union
\end{verbatim}

\begin{Shaded}
\begin{Highlighting}[]
\FunctionTok{library}\NormalTok{(skimr)}
\FunctionTok{library}\NormalTok{(GGally)}
\end{Highlighting}
\end{Shaded}

\begin{verbatim}
## Registered S3 method overwritten by 'GGally':
##   method from   
##   +.gg   ggplot2
\end{verbatim}

\begin{Shaded}
\begin{Highlighting}[]
\FunctionTok{library}\NormalTok{(Hmisc)}
\end{Highlighting}
\end{Shaded}

\begin{verbatim}
## Loading required package: lattice
## Loading required package: survival
## Loading required package: Formula
## 
## Attaching package: 'Hmisc'
## 
## The following objects are masked from 'package:dplyr':
## 
##     src, summarize
## 
## The following objects are masked from 'package:base':
## 
##     format.pval, units
\end{verbatim}

\begin{Shaded}
\begin{Highlighting}[]
\FunctionTok{library}\NormalTok{(RColorBrewer)}
\FunctionTok{library}\NormalTok{(ggpubr)}
\end{Highlighting}
\end{Shaded}

\hypertarget{about-the-dataset}{%
\paragraph{About the dataset}\label{about-the-dataset}}

The product of focus will be Leaf, Bellabeat's classic wellness tracker
that can be worn as a bracelet, necklace, or clip. The Leaf tracker
connects to the Bellabeat app to track activity, sleep, and stress. We
will use FitBit Data gathered from
\href{https://www.kaggle.com/datasets/arashnic/fitbit}{FitBit Fitness
Tracker Data} (CC0: Public Domain, dataset made available through
Mobius). This dataset was generated by respondents to a distributed
survey via Amazon Mechanical Turk between 03.12.2016-05.12.2016.

This Kaggle data set contains personal fitness tracker from thirty
fitbit users. Thirty eligible Fitbit users consented to the submission
of personal tracker data, including minute-level output for physical
activity, heart rate, and sleep monitoring. We will apply trends found
with this dataset to Leaf marketing strategy.

\hypertarget{data-storage-organization-and-verification}{%
\paragraph{Data storage, organization, and
verification}\label{data-storage-organization-and-verification}}

Each dataset was stored as a Microsoft Excel CSV file. Almost all of the
data could be classified as long; participants were repeated for every
time that was recorded, allowing each participant to have multiple rows.
Participants were anonymized with unique ID numbers.

We decided to use the following tables:

\begin{itemize}
\tightlist
\item
  \textbf{Daily Activity} 33 participants for 31 days tracking daily
  steps, distance, minutes, steps, and calories
\item
  \textbf{Hourly Calories} 33 participants for 31 days tracking hourly
  calories burned
\item
  \textbf{Hourly Intensities} 33 participants for 31 days tracking
  hourly total and average intensity
\item
  \textbf{Daily Sleep} 24 participants for 31 days tracking daily
  minutes asleep and in bed
\item
  \textbf{Hourly Steps} 33 participants for 31 days tracking hourly step
  totals
\end{itemize}

\hypertarget{importing-dataset}{%
\paragraph{Importing dataset}\label{importing-dataset}}

\begin{Shaded}
\begin{Highlighting}[]
\NormalTok{daily\_activity }\OtherTok{\textless{}{-}} \FunctionTok{read\_csv}\NormalTok{(}\StringTok{"Fitabase Data 4.12.16{-}5.12.16/dailyActivity\_merged.csv"}\NormalTok{)}
\end{Highlighting}
\end{Shaded}

\begin{verbatim}
## Rows: 940 Columns: 15
## -- Column specification --------------------------------------------------------
## Delimiter: ","
## chr  (1): ActivityDate
## dbl (14): Id, TotalSteps, TotalDistance, TrackerDistance, LoggedActivitiesDi...
## 
## i Use `spec()` to retrieve the full column specification for this data.
## i Specify the column types or set `show_col_types = FALSE` to quiet this message.
\end{verbatim}

\begin{Shaded}
\begin{Highlighting}[]
\NormalTok{day\_sleep }\OtherTok{\textless{}{-}} \FunctionTok{read\_csv}\NormalTok{(}\StringTok{"sleepDay\_merged.csv"}\NormalTok{)}
\end{Highlighting}
\end{Shaded}

\begin{verbatim}
## Rows: 413 Columns: 5
## -- Column specification --------------------------------------------------------
## Delimiter: ","
## chr (1): SleepDay
## dbl (4): Id, TotalSleepRecords, TotalMinutesAsleep, TotalTimeInBed
## 
## i Use `spec()` to retrieve the full column specification for this data.
## i Specify the column types or set `show_col_types = FALSE` to quiet this message.
\end{verbatim}

\begin{Shaded}
\begin{Highlighting}[]
\NormalTok{hour\_calories }\OtherTok{\textless{}{-}} \FunctionTok{read\_csv}\NormalTok{(}\StringTok{"Fitabase Data 4.12.16{-}5.12.16/hourlyCalories\_merged.csv"}\NormalTok{)}
\end{Highlighting}
\end{Shaded}

\begin{verbatim}
## Rows: 22099 Columns: 3
## -- Column specification --------------------------------------------------------
## Delimiter: ","
## chr (1): ActivityHour
## dbl (2): Id, Calories
## 
## i Use `spec()` to retrieve the full column specification for this data.
## i Specify the column types or set `show_col_types = FALSE` to quiet this message.
\end{verbatim}

\begin{Shaded}
\begin{Highlighting}[]
\NormalTok{hour\_intensities }\OtherTok{\textless{}{-}} \FunctionTok{read\_csv}\NormalTok{(}\StringTok{"Fitabase Data 4.12.16{-}5.12.16/hourlyIntensities\_merged.csv"}\NormalTok{)}
\end{Highlighting}
\end{Shaded}

\begin{verbatim}
## Rows: 22099 Columns: 4
## -- Column specification --------------------------------------------------------
## Delimiter: ","
## chr (1): ActivityHour
## dbl (3): Id, TotalIntensity, AverageIntensity
## 
## i Use `spec()` to retrieve the full column specification for this data.
## i Specify the column types or set `show_col_types = FALSE` to quiet this message.
\end{verbatim}

\begin{Shaded}
\begin{Highlighting}[]
\NormalTok{hour\_steps }\OtherTok{\textless{}{-}} \FunctionTok{read\_csv}\NormalTok{(}\StringTok{"Fitabase Data 4.12.16{-}5.12.16/hourlySteps\_merged.csv"}\NormalTok{)}
\end{Highlighting}
\end{Shaded}

\begin{verbatim}
## Rows: 22099 Columns: 3
## -- Column specification --------------------------------------------------------
## Delimiter: ","
## chr (1): ActivityHour
## dbl (2): Id, StepTotal
## 
## i Use `spec()` to retrieve the full column specification for this data.
## i Specify the column types or set `show_col_types = FALSE` to quiet this message.
\end{verbatim}

\hypertarget{data-integrity-and-limitations}{%
\paragraph{Data integrity and
limitations}\label{data-integrity-and-limitations}}

There were a few limitations that may threaten the credibility and
integrity of the data. The data appears both reliable and original. The
data is comprehensive in its measurement of different time
specifications (from the day all the way to the second); however it is
missing demographic data, has the bare minimum in sample size, and only
lasts for two months. This may introduce sampling bias as the data 1)
may not be representative of the population and 2) does not account for
difference in trends outside of April and May. The data is also not
current, ending almost 7 years from the time of this case study (2016).

Although certain datasets was desirable for analysis, they were excluded
as their validity was questionable. For instance, the WeightLogInfo
dataset only included 8 participants; most only had one or two
responses, making tracking progress impossible.

We can progress with caution to process our data.

\hypertarget{process}{%
\subsection{Process}\label{process}}

\hypertarget{checking-datasets}{%
\paragraph{Checking datasets}\label{checking-datasets}}

First we took a look at the datasets and verified the number of distinct
participants.

\begin{Shaded}
\begin{Highlighting}[]
\FunctionTok{glimpse}\NormalTok{(daily\_activity)}
\end{Highlighting}
\end{Shaded}

\begin{verbatim}
## Rows: 940
## Columns: 15
## $ Id                       <dbl> 1503960366, 1503960366, 1503960366, 150396036~
## $ ActivityDate             <chr> "4/12/2016", "4/13/2016", "4/14/2016", "4/15/~
## $ TotalSteps               <dbl> 13162, 10735, 10460, 9762, 12669, 9705, 13019~
## $ TotalDistance            <dbl> 8.50, 6.97, 6.74, 6.28, 8.16, 6.48, 8.59, 9.8~
## $ TrackerDistance          <dbl> 8.50, 6.97, 6.74, 6.28, 8.16, 6.48, 8.59, 9.8~
## $ LoggedActivitiesDistance <dbl> 0, 0, 0, 0, 0, 0, 0, 0, 0, 0, 0, 0, 0, 0, 0, ~
## $ VeryActiveDistance       <dbl> 1.88, 1.57, 2.44, 2.14, 2.71, 3.19, 3.25, 3.5~
## $ ModeratelyActiveDistance <dbl> 0.55, 0.69, 0.40, 1.26, 0.41, 0.78, 0.64, 1.3~
## $ LightActiveDistance      <dbl> 6.06, 4.71, 3.91, 2.83, 5.04, 2.51, 4.71, 5.0~
## $ SedentaryActiveDistance  <dbl> 0, 0, 0, 0, 0, 0, 0, 0, 0, 0, 0, 0, 0, 0, 0, ~
## $ VeryActiveMinutes        <dbl> 25, 21, 30, 29, 36, 38, 42, 50, 28, 19, 66, 4~
## $ FairlyActiveMinutes      <dbl> 13, 19, 11, 34, 10, 20, 16, 31, 12, 8, 27, 21~
## $ LightlyActiveMinutes     <dbl> 328, 217, 181, 209, 221, 164, 233, 264, 205, ~
## $ SedentaryMinutes         <dbl> 728, 776, 1218, 726, 773, 539, 1149, 775, 818~
## $ Calories                 <dbl> 1985, 1797, 1776, 1745, 1863, 1728, 1921, 203~
\end{verbatim}

\begin{Shaded}
\begin{Highlighting}[]
\FunctionTok{n\_distinct}\NormalTok{(daily\_activity}\SpecialCharTok{$}\NormalTok{Id)}
\end{Highlighting}
\end{Shaded}

\begin{verbatim}
## [1] 33
\end{verbatim}

\begin{Shaded}
\begin{Highlighting}[]
\FunctionTok{glimpse}\NormalTok{(day\_sleep)}
\end{Highlighting}
\end{Shaded}

\begin{verbatim}
## Rows: 413
## Columns: 5
## $ Id                 <dbl> 1503960366, 1503960366, 1503960366, 1503960366, 150~
## $ SleepDay           <chr> "4/12/2016 12:00:00 AM", "4/13/2016 12:00:00 AM", "~
## $ TotalSleepRecords  <dbl> 1, 2, 1, 2, 1, 1, 1, 1, 1, 1, 1, 1, 1, 1, 1, 1, 1, ~
## $ TotalMinutesAsleep <dbl> 327, 384, 412, 340, 700, 304, 360, 325, 361, 430, 2~
## $ TotalTimeInBed     <dbl> 346, 407, 442, 367, 712, 320, 377, 364, 384, 449, 3~
\end{verbatim}

\begin{Shaded}
\begin{Highlighting}[]
\FunctionTok{n\_distinct}\NormalTok{(day\_sleep}\SpecialCharTok{$}\NormalTok{Id)}
\end{Highlighting}
\end{Shaded}

\begin{verbatim}
## [1] 24
\end{verbatim}

\begin{Shaded}
\begin{Highlighting}[]
\FunctionTok{glimpse}\NormalTok{(hour\_calories)}
\end{Highlighting}
\end{Shaded}

\begin{verbatim}
## Rows: 22,099
## Columns: 3
## $ Id           <dbl> 1503960366, 1503960366, 1503960366, 1503960366, 150396036~
## $ ActivityHour <chr> "4/12/2016 12:00:00 AM", "4/12/2016 1:00:00 AM", "4/12/20~
## $ Calories     <dbl> 81, 61, 59, 47, 48, 48, 48, 47, 68, 141, 99, 76, 73, 66, ~
\end{verbatim}

\begin{Shaded}
\begin{Highlighting}[]
\FunctionTok{n\_distinct}\NormalTok{(hour\_calories}\SpecialCharTok{$}\NormalTok{Id)}
\end{Highlighting}
\end{Shaded}

\begin{verbatim}
## [1] 33
\end{verbatim}

\begin{Shaded}
\begin{Highlighting}[]
\FunctionTok{glimpse}\NormalTok{(hour\_intensities}\SpecialCharTok{$}\NormalTok{Id)}
\end{Highlighting}
\end{Shaded}

\begin{verbatim}
##  num [1:22099] 1.5e+09 1.5e+09 1.5e+09 1.5e+09 1.5e+09 ...
\end{verbatim}

\begin{Shaded}
\begin{Highlighting}[]
\FunctionTok{n\_distinct}\NormalTok{(hour\_intensities}\SpecialCharTok{$}\NormalTok{Id)}
\end{Highlighting}
\end{Shaded}

\begin{verbatim}
## [1] 33
\end{verbatim}

\begin{Shaded}
\begin{Highlighting}[]
\FunctionTok{glimpse}\NormalTok{(hour\_steps}\SpecialCharTok{$}\NormalTok{Id)}
\end{Highlighting}
\end{Shaded}

\begin{verbatim}
##  num [1:22099] 1.5e+09 1.5e+09 1.5e+09 1.5e+09 1.5e+09 ...
\end{verbatim}

\begin{Shaded}
\begin{Highlighting}[]
\FunctionTok{n\_distinct}\NormalTok{(hour\_steps}\SpecialCharTok{$}\NormalTok{Id)}
\end{Highlighting}
\end{Shaded}

\begin{verbatim}
## [1] 33
\end{verbatim}

\hypertarget{cleaning-data}{%
\paragraph{Cleaning data}\label{cleaning-data}}

First we used the clean\_names() function to make the column names more
uniform and the colnames() function to check the changes.

\begin{Shaded}
\begin{Highlighting}[]
\NormalTok{daily\_activity }\OtherTok{\textless{}{-}}\NormalTok{ daily\_activity }\SpecialCharTok{\%\textgreater{}\%}
  \FunctionTok{clean\_names}\NormalTok{() }\SpecialCharTok{\%\textgreater{}\%}
  \FunctionTok{rename}\NormalTok{(}\AttributeTok{date =}\NormalTok{ activity\_date)}
\FunctionTok{colnames}\NormalTok{(daily\_activity)}
\end{Highlighting}
\end{Shaded}

\begin{verbatim}
##  [1] "id"                         "date"                      
##  [3] "total_steps"                "total_distance"            
##  [5] "tracker_distance"           "logged_activities_distance"
##  [7] "very_active_distance"       "moderately_active_distance"
##  [9] "light_active_distance"      "sedentary_active_distance" 
## [11] "very_active_minutes"        "fairly_active_minutes"     
## [13] "lightly_active_minutes"     "sedentary_minutes"         
## [15] "calories"
\end{verbatim}

\begin{Shaded}
\begin{Highlighting}[]
\NormalTok{day\_sleep }\OtherTok{\textless{}{-}}\NormalTok{ day\_sleep }\SpecialCharTok{\%\textgreater{}\%}
  \FunctionTok{clean\_names}\NormalTok{() }\SpecialCharTok{\%\textgreater{}\%}
  \FunctionTok{rename}\NormalTok{(}\AttributeTok{date =}\NormalTok{ sleep\_day)}
\FunctionTok{colnames}\NormalTok{(day\_sleep)}
\end{Highlighting}
\end{Shaded}

\begin{verbatim}
## [1] "id"                   "date"                 "total_sleep_records" 
## [4] "total_minutes_asleep" "total_time_in_bed"
\end{verbatim}

\begin{Shaded}
\begin{Highlighting}[]
\NormalTok{hour\_calories }\OtherTok{\textless{}{-}} \FunctionTok{clean\_names}\NormalTok{(hour\_calories)}
\FunctionTok{colnames}\NormalTok{(hour\_calories)}
\end{Highlighting}
\end{Shaded}

\begin{verbatim}
## [1] "id"            "activity_hour" "calories"
\end{verbatim}

\begin{Shaded}
\begin{Highlighting}[]
\NormalTok{hour\_intensities }\OtherTok{\textless{}{-}} \FunctionTok{clean\_names}\NormalTok{(hour\_intensities)}
\FunctionTok{colnames}\NormalTok{(hour\_intensities)}
\end{Highlighting}
\end{Shaded}

\begin{verbatim}
## [1] "id"                "activity_hour"     "total_intensity"  
## [4] "average_intensity"
\end{verbatim}

\begin{Shaded}
\begin{Highlighting}[]
\NormalTok{hour\_steps }\OtherTok{\textless{}{-}} \FunctionTok{clean\_names}\NormalTok{(hour\_steps)}
\FunctionTok{colnames}\NormalTok{(hour\_steps)}
\end{Highlighting}
\end{Shaded}

\begin{verbatim}
## [1] "id"            "activity_hour" "step_total"
\end{verbatim}

We also found that the variables id and date were not in the right data
format, so we fixed them accordingly. We turned id into a string since
each id represented individual people and not numerical integers.

\begin{Shaded}
\begin{Highlighting}[]
\NormalTok{daily\_activity }\OtherTok{\textless{}{-}}\NormalTok{ daily\_activity }\SpecialCharTok{\%\textgreater{}\%} 
  \FunctionTok{mutate}\NormalTok{(}\AttributeTok{activity\_date =} \FunctionTok{mdy}\NormalTok{(daily\_activity}\SpecialCharTok{$}\NormalTok{date),}
         \AttributeTok{id =} \FunctionTok{as.character}\NormalTok{(id))}

\NormalTok{day\_sleep }\OtherTok{\textless{}{-}}\NormalTok{ day\_sleep }\SpecialCharTok{\%\textgreater{}\%}
  \FunctionTok{mutate}\NormalTok{(}\AttributeTok{sleep\_day =} \FunctionTok{mdy\_hms}\NormalTok{(day\_sleep}\SpecialCharTok{$}\NormalTok{date),}
         \AttributeTok{id =} \FunctionTok{as.character}\NormalTok{(id))}

\NormalTok{hour\_calories }\OtherTok{\textless{}{-}}\NormalTok{ hour\_calories }\SpecialCharTok{\%\textgreater{}\%}
  \FunctionTok{mutate}\NormalTok{(}\AttributeTok{activity\_hour =} \FunctionTok{mdy\_hms}\NormalTok{(activity\_hour),}
         \AttributeTok{id =} \FunctionTok{as.character}\NormalTok{((id)))}

\NormalTok{hour\_intensities }\OtherTok{\textless{}{-}}\NormalTok{ hour\_intensities }\SpecialCharTok{\%\textgreater{}\%}
  \FunctionTok{mutate}\NormalTok{(}\AttributeTok{activity\_hour =} \FunctionTok{mdy\_hms}\NormalTok{(activity\_hour),}
         \AttributeTok{id =} \FunctionTok{as.character}\NormalTok{(id))}

\NormalTok{hour\_steps }\OtherTok{\textless{}{-}}\NormalTok{ hour\_steps }\SpecialCharTok{\%\textgreater{}\%}
  \FunctionTok{mutate}\NormalTok{(}\AttributeTok{activity\_hour =} \FunctionTok{mdy\_hms}\NormalTok{(activity\_hour),}
         \AttributeTok{id =} \FunctionTok{as.character}\NormalTok{(id))}
\end{Highlighting}
\end{Shaded}

Then, we found that day\_sleep was the only dataset with duplicate rows,
so we removed them.

\begin{Shaded}
\begin{Highlighting}[]
\FunctionTok{sum}\NormalTok{(}\FunctionTok{duplicated}\NormalTok{(daily\_activity))}
\end{Highlighting}
\end{Shaded}

\begin{verbatim}
## [1] 0
\end{verbatim}

\begin{Shaded}
\begin{Highlighting}[]
\FunctionTok{sum}\NormalTok{(}\FunctionTok{duplicated}\NormalTok{(day\_sleep))}
\end{Highlighting}
\end{Shaded}

\begin{verbatim}
## [1] 3
\end{verbatim}

\begin{Shaded}
\begin{Highlighting}[]
\FunctionTok{sum}\NormalTok{(}\FunctionTok{duplicated}\NormalTok{(hour\_calories))}
\end{Highlighting}
\end{Shaded}

\begin{verbatim}
## [1] 0
\end{verbatim}

\begin{Shaded}
\begin{Highlighting}[]
\FunctionTok{sum}\NormalTok{(}\FunctionTok{duplicated}\NormalTok{(hour\_intensities))}
\end{Highlighting}
\end{Shaded}

\begin{verbatim}
## [1] 0
\end{verbatim}

\begin{Shaded}
\begin{Highlighting}[]
\FunctionTok{sum}\NormalTok{(}\FunctionTok{duplicated}\NormalTok{(hour\_steps))}
\end{Highlighting}
\end{Shaded}

\begin{verbatim}
## [1] 0
\end{verbatim}

\begin{Shaded}
\begin{Highlighting}[]
\FunctionTok{count}\NormalTok{(day\_sleep)  }\CommentTok{\# number of rows in day\_sleep}
\end{Highlighting}
\end{Shaded}

\begin{verbatim}
## # A tibble: 1 x 1
##       n
##   <int>
## 1   413
\end{verbatim}

\begin{Shaded}
\begin{Highlighting}[]
\NormalTok{day\_sleep }\OtherTok{\textless{}{-}}\NormalTok{ day\_sleep }\SpecialCharTok{\%\textgreater{}\%}
  \FunctionTok{distinct}\NormalTok{() }\SpecialCharTok{\%\textgreater{}\%}
  \FunctionTok{drop\_na}\NormalTok{()}
\FunctionTok{count}\NormalTok{(day\_sleep)}
\end{Highlighting}
\end{Shaded}

\begin{verbatim}
## # A tibble: 1 x 1
##       n
##   <int>
## 1   410
\end{verbatim}

\hypertarget{merging-the-datasets}{%
\paragraph{Merging the datasets}\label{merging-the-datasets}}

Finally, we merged the datasets. First, we verified the number of rows
in each table. We noted that day\_sleep has about half of the number of
rows as daily\_activity.

\begin{Shaded}
\begin{Highlighting}[]
\CommentTok{\# First we verified number of rows}
\FunctionTok{nrow}\NormalTok{(daily\_activity)}
\end{Highlighting}
\end{Shaded}

\begin{verbatim}
## [1] 940
\end{verbatim}

\begin{Shaded}
\begin{Highlighting}[]
\FunctionTok{nrow}\NormalTok{(day\_sleep)}
\end{Highlighting}
\end{Shaded}

\begin{verbatim}
## [1] 410
\end{verbatim}

\begin{Shaded}
\begin{Highlighting}[]
\FunctionTok{nrow}\NormalTok{(hour\_calories)}
\end{Highlighting}
\end{Shaded}

\begin{verbatim}
## [1] 22099
\end{verbatim}

\begin{Shaded}
\begin{Highlighting}[]
\FunctionTok{nrow}\NormalTok{(hour\_intensities)}
\end{Highlighting}
\end{Shaded}

\begin{verbatim}
## [1] 22099
\end{verbatim}

\begin{Shaded}
\begin{Highlighting}[]
\FunctionTok{nrow}\NormalTok{(hour\_steps)}
\end{Highlighting}
\end{Shaded}

\begin{verbatim}
## [1] 22099
\end{verbatim}

Then, we merged the datasets by separating them between the daily and
hourly. We used id as well as date or activity\_hour as the primary
keys.

\begin{Shaded}
\begin{Highlighting}[]
\CommentTok{\# Merging datasets that tracked daily}
\NormalTok{day\_sleep }\OtherTok{\textless{}{-}}\NormalTok{ day\_sleep }\SpecialCharTok{\%\textgreater{}\%}
  \FunctionTok{separate}\NormalTok{(date, }\AttributeTok{into=}\FunctionTok{c}\NormalTok{(}\StringTok{\textquotesingle{}date\textquotesingle{}}\NormalTok{,}\StringTok{\textquotesingle{}time\textquotesingle{}}\NormalTok{), }\AttributeTok{sep=}\StringTok{\textquotesingle{} \textquotesingle{}}\NormalTok{)}
\end{Highlighting}
\end{Shaded}

\begin{verbatim}
## Warning: Expected 2 pieces. Additional pieces discarded in 410 rows [1, 2, 3, 4,
## 5, 6, 7, 8, 9, 10, 11, 12, 13, 14, 15, 16, 17, 18, 19, 20, ...].
\end{verbatim}

\begin{Shaded}
\begin{Highlighting}[]
\NormalTok{daily\_activity\_sleep }\OtherTok{\textless{}{-}} \FunctionTok{merge}\NormalTok{(daily\_activity, day\_sleep, }\AttributeTok{by=}\FunctionTok{c}\NormalTok{(}\StringTok{"id"}\NormalTok{,}\StringTok{"date"}\NormalTok{))}
\FunctionTok{glimpse}\NormalTok{(daily\_activity\_sleep)}
\end{Highlighting}
\end{Shaded}

\begin{verbatim}
## Rows: 410
## Columns: 21
## $ id                         <chr> "1503960366", "1503960366", "1503960366", "~
## $ date                       <chr> "4/12/2016", "4/13/2016", "4/15/2016", "4/1~
## $ total_steps                <dbl> 13162, 10735, 9762, 12669, 9705, 15506, 105~
## $ total_distance             <dbl> 8.50, 6.97, 6.28, 8.16, 6.48, 9.88, 6.68, 6~
## $ tracker_distance           <dbl> 8.50, 6.97, 6.28, 8.16, 6.48, 9.88, 6.68, 6~
## $ logged_activities_distance <dbl> 0, 0, 0, 0, 0, 0, 0, 0, 0, 0, 0, 0, 0, 0, 0~
## $ very_active_distance       <dbl> 1.88, 1.57, 2.14, 2.71, 3.19, 3.53, 1.96, 1~
## $ moderately_active_distance <dbl> 0.55, 0.69, 1.26, 0.41, 0.78, 1.32, 0.48, 0~
## $ light_active_distance      <dbl> 6.06, 4.71, 2.83, 5.04, 2.51, 5.03, 4.24, 4~
## $ sedentary_active_distance  <dbl> 0, 0, 0, 0, 0, 0, 0, 0, 0, 0, 0, 0, 0, 0, 0~
## $ very_active_minutes        <dbl> 25, 21, 29, 36, 38, 50, 28, 19, 41, 39, 73,~
## $ fairly_active_minutes      <dbl> 13, 19, 34, 10, 20, 31, 12, 8, 21, 5, 14, 2~
## $ lightly_active_minutes     <dbl> 328, 217, 209, 221, 164, 264, 205, 211, 262~
## $ sedentary_minutes          <dbl> 728, 776, 726, 773, 539, 775, 818, 838, 732~
## $ calories                   <dbl> 1985, 1797, 1745, 1863, 1728, 2035, 1786, 1~
## $ activity_date              <date> 2016-04-12, 2016-04-13, 2016-04-15, 2016-0~
## $ time                       <chr> "12:00:00", "12:00:00", "12:00:00", "12:00:~
## $ total_sleep_records        <dbl> 1, 2, 1, 2, 1, 1, 1, 1, 1, 1, 1, 1, 1, 1, 1~
## $ total_minutes_asleep       <dbl> 327, 384, 412, 340, 700, 304, 360, 325, 361~
## $ total_time_in_bed          <dbl> 346, 407, 442, 367, 712, 320, 377, 364, 384~
## $ sleep_day                  <dttm> 2016-04-12, 2016-04-13, 2016-04-15, 2016-0~
\end{verbatim}

\begin{Shaded}
\begin{Highlighting}[]
\FunctionTok{nrow}\NormalTok{(daily\_activity\_sleep)}
\end{Highlighting}
\end{Shaded}

\begin{verbatim}
## [1] 410
\end{verbatim}

\begin{Shaded}
\begin{Highlighting}[]
\CommentTok{\# Merging datasets that tracked hourly}
\NormalTok{hourly\_activity }\OtherTok{\textless{}{-}} \FunctionTok{list}\NormalTok{(hour\_calories, hour\_intensities, hour\_steps) }\SpecialCharTok{\%\textgreater{}\%}
  \FunctionTok{reduce}\NormalTok{(full\_join, }\AttributeTok{by=}\FunctionTok{c}\NormalTok{(}\StringTok{"id"}\NormalTok{,}\StringTok{"activity\_hour"}\NormalTok{))}
\FunctionTok{glimpse}\NormalTok{(hourly\_activity)}
\end{Highlighting}
\end{Shaded}

\begin{verbatim}
## Rows: 22,099
## Columns: 6
## $ id                <chr> "1503960366", "1503960366", "1503960366", "150396036~
## $ activity_hour     <dttm> 2016-04-12 00:00:00, 2016-04-12 01:00:00, 2016-04-1~
## $ calories          <dbl> 81, 61, 59, 47, 48, 48, 48, 47, 68, 141, 99, 76, 73,~
## $ total_intensity   <dbl> 20, 8, 7, 0, 0, 0, 0, 0, 13, 30, 29, 12, 11, 6, 36, ~
## $ average_intensity <dbl> 0.333333, 0.133333, 0.116667, 0.000000, 0.000000, 0.~
## $ step_total        <dbl> 373, 160, 151, 0, 0, 0, 0, 0, 250, 1864, 676, 360, 2~
\end{verbatim}

\begin{Shaded}
\begin{Highlighting}[]
\FunctionTok{nrow}\NormalTok{(hourly\_activity)}
\end{Highlighting}
\end{Shaded}

\begin{verbatim}
## [1] 22099
\end{verbatim}

\hypertarget{analyze-and-share}{%
\subsection{Analyze and Share}\label{analyze-and-share}}

\hypertarget{counting-the-hours}{%
\paragraph{Counting the hours}\label{counting-the-hours}}

Firstly, we created a dataframe by adding a column in the merged
hourly\_activity dataset to keep track of the hours.

\begin{Shaded}
\begin{Highlighting}[]
\NormalTok{hours }\OtherTok{\textless{}{-}}\NormalTok{ hourly\_activity }\SpecialCharTok{\%\textgreater{}\%}
  \FunctionTok{mutate}\NormalTok{(}\AttributeTok{time =} \FunctionTok{hour}\NormalTok{(activity\_hour)) }\SpecialCharTok{\%\textgreater{}\%}
  \FunctionTok{group\_by}\NormalTok{(time) }\SpecialCharTok{\%\textgreater{}\%}
  \FunctionTok{arrange}\NormalTok{(time)}
\end{Highlighting}
\end{Shaded}

Using this, we wanted to look at the correlations between the variables
measured using the rcorr() function from the ``Hmisc'' package, which
also allowed us to look at the significance of the correlation.

\begin{Shaded}
\begin{Highlighting}[]
\NormalTok{hours.rcorr }\OtherTok{\textless{}{-}}\NormalTok{ hours }\SpecialCharTok{\%\textgreater{}\%}
  \FunctionTok{select}\NormalTok{(time,calories,average\_intensity,step\_total) }\SpecialCharTok{\%\textgreater{}\%}
  \FunctionTok{as.matrix}\NormalTok{() }\SpecialCharTok{\%\textgreater{}\%}
  \FunctionTok{rcorr}\NormalTok{()}
\NormalTok{hours.rcorr}
\end{Highlighting}
\end{Shaded}

\begin{verbatim}
##                   time calories average_intensity step_total
## time              1.00     0.18              0.21       0.18
## calories          0.18     1.00              0.90       0.81
## average_intensity 0.21     0.90              1.00       0.90
## step_total        0.18     0.81              0.90       1.00
## 
## n= 22099 
## 
## 
## P
##                   time calories average_intensity step_total
## time                    0        0                 0        
## calories           0             0                 0        
## average_intensity  0    0                          0        
## step_total         0    0        0
\end{verbatim}

\begin{Shaded}
\begin{Highlighting}[]
\NormalTok{hours.rcorr}\SpecialCharTok{$}\NormalTok{P }\SpecialCharTok{\textless{}}\NormalTok{ .}\DecValTok{01}
\end{Highlighting}
\end{Shaded}

\begin{verbatim}
##                   time calories average_intensity step_total
## time                NA     TRUE              TRUE       TRUE
## calories          TRUE       NA              TRUE       TRUE
## average_intensity TRUE     TRUE                NA       TRUE
## step_total        TRUE     TRUE              TRUE         NA
\end{verbatim}

We saw a strong significant correlation between calories, average
intensity, and step total. We wanted to visualize this.

\begin{Shaded}
\begin{Highlighting}[]
\CommentTok{\# step\_total by calories}
\NormalTok{hours }\SpecialCharTok{\%\textgreater{}\%}
  \FunctionTok{ggplot}\NormalTok{() }\SpecialCharTok{+} 
  \FunctionTok{geom\_point}\NormalTok{(}\AttributeTok{mapping =} \FunctionTok{aes}\NormalTok{(}\AttributeTok{x=}\NormalTok{step\_total,}\AttributeTok{y=}\NormalTok{calories,}\AttributeTok{color=}\NormalTok{average\_intensity,}\AttributeTok{size=}\NormalTok{average\_intensity)) }\SpecialCharTok{+} 
  \FunctionTok{geom\_smooth}\NormalTok{(}\AttributeTok{mapping =} \FunctionTok{aes}\NormalTok{(}\AttributeTok{x=}\NormalTok{step\_total,}\AttributeTok{y=}\NormalTok{calories,}\AttributeTok{color=}\NormalTok{average\_intensity), }\AttributeTok{alpha =} \DecValTok{0}\NormalTok{, }\AttributeTok{color =} \StringTok{"\#d7301f"}\NormalTok{) }\SpecialCharTok{+} 
  \FunctionTok{scale\_color\_gradient}\NormalTok{(}\AttributeTok{low =} \StringTok{"\#fee8c8"}\NormalTok{, }\AttributeTok{high =} \StringTok{"\#d7301f"}\NormalTok{) }\SpecialCharTok{+}
  \FunctionTok{guides}\NormalTok{(}\AttributeTok{color =} \StringTok{\textquotesingle{}legend\textquotesingle{}}\NormalTok{) }\SpecialCharTok{+}
  \FunctionTok{theme}\NormalTok{(}\AttributeTok{plot.background =} \FunctionTok{element\_rect}\NormalTok{(}\AttributeTok{fill =} \StringTok{"transparent"}\NormalTok{, }\AttributeTok{color =} \StringTok{"transparent"}\NormalTok{),}
        \AttributeTok{panel.background =} \FunctionTok{element\_rect}\NormalTok{(}\AttributeTok{fill =} \StringTok{"transparent"}\NormalTok{, }\AttributeTok{colour =} \StringTok{"gray"}\NormalTok{),}
        \AttributeTok{panel.border =} \FunctionTok{element\_rect}\NormalTok{(}\AttributeTok{fill =} \StringTok{"transparent"}\NormalTok{, }\AttributeTok{colour =} \StringTok{"black"}\NormalTok{),}
        \AttributeTok{panel.grid.major =} \FunctionTok{element\_line}\NormalTok{(}\AttributeTok{colour =} \StringTok{"grey"}\NormalTok{, }\AttributeTok{linetype =} \StringTok{"dashed"}\NormalTok{),}
        \AttributeTok{legend.direction =} \StringTok{"horizontal"}\NormalTok{,}
        \AttributeTok{legend.position =} \FunctionTok{c}\NormalTok{(.}\DecValTok{77}\NormalTok{,.}\DecValTok{15}\NormalTok{), }
        \AttributeTok{legend.background =} \FunctionTok{element\_blank}\NormalTok{()) }\SpecialCharTok{+}
  \FunctionTok{labs}\NormalTok{(}\AttributeTok{title =} \StringTok{"Calories by step total"}\NormalTok{, }\AttributeTok{subtitle =} \StringTok{"Colored by average intensity"}\NormalTok{)}
\end{Highlighting}
\end{Shaded}

\begin{verbatim}
## `geom_smooth()` using method = 'gam' and formula 'y ~ s(x, bs = "cs")'
\end{verbatim}

\includegraphics{Case-Study-Bellabeat_files/figure-latex/unnamed-chunk-1-1.pdf}

\begin{Shaded}
\begin{Highlighting}[]
\NormalTok{hours }\SpecialCharTok{\%\textgreater{}\%}
  \FunctionTok{ggplot}\NormalTok{() }\SpecialCharTok{+} 
  \FunctionTok{geom\_point}\NormalTok{(}\AttributeTok{mapping =} \FunctionTok{aes}\NormalTok{(}\AttributeTok{x=}\NormalTok{step\_total,}\AttributeTok{y=}\NormalTok{average\_intensity,}\AttributeTok{color=}\NormalTok{calories,}\AttributeTok{size=}\NormalTok{calories)) }\SpecialCharTok{+} 
  \FunctionTok{geom\_smooth}\NormalTok{(}\AttributeTok{mapping =} \FunctionTok{aes}\NormalTok{(}\AttributeTok{x=}\NormalTok{step\_total,}\AttributeTok{y=}\NormalTok{average\_intensity,}\AttributeTok{color=}\NormalTok{calories), }\AttributeTok{alpha =} \DecValTok{0}\NormalTok{,}\AttributeTok{color =} \StringTok{"blue"}\NormalTok{) }\SpecialCharTok{+} 
  \FunctionTok{scale\_color\_gradient}\NormalTok{(}\AttributeTok{low =} \StringTok{"\#f7fcfd"}\NormalTok{, }\AttributeTok{high =} \StringTok{"blue"}\NormalTok{) }\SpecialCharTok{+}
  \FunctionTok{guides}\NormalTok{(}\AttributeTok{color =} \StringTok{\textquotesingle{}legend\textquotesingle{}}\NormalTok{) }\SpecialCharTok{+}
  \FunctionTok{theme}\NormalTok{(}\AttributeTok{plot.background =} \FunctionTok{element\_rect}\NormalTok{(}\AttributeTok{fill =} \StringTok{"transparent"}\NormalTok{, }\AttributeTok{color =} \StringTok{"transparent"}\NormalTok{),}
        \AttributeTok{panel.background =} \FunctionTok{element\_rect}\NormalTok{(}\AttributeTok{fill =} \StringTok{"transparent"}\NormalTok{, }\AttributeTok{colour =} \StringTok{"gray"}\NormalTok{),}
        \AttributeTok{panel.border =} \FunctionTok{element\_rect}\NormalTok{(}\AttributeTok{fill =} \StringTok{"transparent"}\NormalTok{, }\AttributeTok{colour =} \StringTok{"black"}\NormalTok{),}
        \AttributeTok{panel.grid.major =} \FunctionTok{element\_line}\NormalTok{(}\AttributeTok{colour =} \StringTok{"grey"}\NormalTok{, }\AttributeTok{linetype =} \StringTok{"dashed"}\NormalTok{),}
        \AttributeTok{legend.direction =} \StringTok{"horizontal"}\NormalTok{,}
        \AttributeTok{legend.position =} \FunctionTok{c}\NormalTok{(.}\DecValTok{75}\NormalTok{,.}\DecValTok{30}\NormalTok{), }
        \AttributeTok{legend.background =} \FunctionTok{element\_blank}\NormalTok{()) }\SpecialCharTok{+}
  \FunctionTok{labs}\NormalTok{(}\AttributeTok{title =} \StringTok{"Step total by average intensity"}\NormalTok{, }\AttributeTok{subtitle =} \StringTok{"Colored by calories"}\NormalTok{)}
\end{Highlighting}
\end{Shaded}

\begin{verbatim}
## `geom_smooth()` using method = 'gam' and formula 'y ~ s(x, bs = "cs")'
\end{verbatim}

\includegraphics{Case-Study-Bellabeat_files/figure-latex/unnamed-chunk-1-2.pdf}

\begin{Shaded}
\begin{Highlighting}[]
\NormalTok{hours }\SpecialCharTok{\%\textgreater{}\%}
  \FunctionTok{ggplot}\NormalTok{() }\SpecialCharTok{+} 
  \FunctionTok{geom\_point}\NormalTok{(}\AttributeTok{mapping =} \FunctionTok{aes}\NormalTok{(}\AttributeTok{x=}\NormalTok{calories,}\AttributeTok{y=}\NormalTok{average\_intensity,}\AttributeTok{color=}\NormalTok{step\_total,}\AttributeTok{size=}\NormalTok{step\_total)) }\SpecialCharTok{+} 
  \FunctionTok{geom\_smooth}\NormalTok{(}\AttributeTok{mapping =} \FunctionTok{aes}\NormalTok{(}\AttributeTok{x=}\NormalTok{calories,}\AttributeTok{y=}\NormalTok{average\_intensity,}\AttributeTok{color=}\NormalTok{step\_total), }\AttributeTok{alpha =} \DecValTok{0}\NormalTok{, }\AttributeTok{color =} \StringTok{"\#238b45"}\NormalTok{) }\SpecialCharTok{+} 
  \FunctionTok{scale\_color\_gradient}\NormalTok{(}\AttributeTok{low =} \StringTok{"\#e5f5e0"}\NormalTok{, }\AttributeTok{high =} \StringTok{"\#006d2c"}\NormalTok{) }\SpecialCharTok{+}
  \FunctionTok{guides}\NormalTok{(}\AttributeTok{color =} \StringTok{\textquotesingle{}legend\textquotesingle{}}\NormalTok{) }\SpecialCharTok{+}
  \FunctionTok{theme}\NormalTok{(}\AttributeTok{plot.background =} \FunctionTok{element\_rect}\NormalTok{(}\AttributeTok{fill =} \StringTok{"transparent"}\NormalTok{, }\AttributeTok{color =} \StringTok{"transparent"}\NormalTok{),}
        \AttributeTok{panel.background =} \FunctionTok{element\_rect}\NormalTok{(}\AttributeTok{fill =} \StringTok{"transparent"}\NormalTok{, }\AttributeTok{colour =} \StringTok{"grey"}\NormalTok{),}
        \AttributeTok{panel.border =} \FunctionTok{element\_rect}\NormalTok{(}\AttributeTok{fill =} \StringTok{"transparent"}\NormalTok{, }\AttributeTok{colour =} \StringTok{"black"}\NormalTok{),}
        \AttributeTok{panel.grid.major =} \FunctionTok{element\_line}\NormalTok{(}\AttributeTok{colour =} \StringTok{"grey"}\NormalTok{, }\AttributeTok{linetype =} \StringTok{"dashed"}\NormalTok{),}
        \AttributeTok{legend.direction =} \StringTok{"horizontal"}\NormalTok{,}
        \AttributeTok{legend.position =} \FunctionTok{c}\NormalTok{(.}\DecValTok{7}\NormalTok{,.}\DecValTok{30}\NormalTok{), }
        \AttributeTok{legend.background =} \FunctionTok{element\_blank}\NormalTok{()) }\SpecialCharTok{+}
  \FunctionTok{labs}\NormalTok{(}\AttributeTok{title =} \StringTok{"Calories by average intensity"}\NormalTok{, }\AttributeTok{subtitle =} \StringTok{"Colored by step total"}\NormalTok{)}
\end{Highlighting}
\end{Shaded}

\begin{verbatim}
## `geom_smooth()` using method = 'gam' and formula 'y ~ s(x, bs = "cs")'
\end{verbatim}

\includegraphics{Case-Study-Bellabeat_files/figure-latex/unnamed-chunk-1-3.pdf}

We can make several observations about these correlational graphs:

\emph{There is a strong significant positive correlation between average
intensity, step total, and calories burned, as expected; the higher any
one of these variables are, the higher any of the other ones are as
well. }The average intensity by calories trend line starts off very
steep. This seems to suggest that relatively low activity intensity
might still have large benefits in terms of burning calories. This could
be an important point to touch on for users as it may persuade more
sedentary users to engage in light, or even moderate, levels of activity
intensity. This means that even walking more than one usually would is
very good.

More hourly stuff

We wanted to see if there were any trends throughout the day with each
passing hour.

\begin{Shaded}
\begin{Highlighting}[]
\FunctionTok{ggplot}\NormalTok{(hours, }\FunctionTok{aes}\NormalTok{(}\AttributeTok{x =}\NormalTok{ time, }\AttributeTok{y =}\NormalTok{ average\_intensity, }\AttributeTok{group =}\NormalTok{ time)) }\SpecialCharTok{+}
  \FunctionTok{geom\_point}\NormalTok{(}\AttributeTok{alpha =} \FloatTok{0.3}\NormalTok{, }\AttributeTok{position =} \StringTok{"jitter"}\NormalTok{, }\AttributeTok{color =} \StringTok{"\#d7301f"}\NormalTok{) }\SpecialCharTok{+}
  \FunctionTok{geom\_boxplot}\NormalTok{(}\AttributeTok{alpha =} \DecValTok{0}\NormalTok{, }\AttributeTok{colour =} \StringTok{"black"}\NormalTok{) }\SpecialCharTok{+}
  \FunctionTok{labs}\NormalTok{(}\AttributeTok{title =} \StringTok{"Average intensity by hour"}\NormalTok{, }\AttributeTok{x=}\StringTok{"Hour"}\NormalTok{, }\AttributeTok{y=}\StringTok{"Average Intensity"}\NormalTok{)}
\end{Highlighting}
\end{Shaded}

\includegraphics{Case-Study-Bellabeat_files/figure-latex/unnamed-chunk-2-1.pdf}

\begin{Shaded}
\begin{Highlighting}[]
\NormalTok{hours }\SpecialCharTok{\%\textgreater{}\%}
  \FunctionTok{group\_by}\NormalTok{(time) }\SpecialCharTok{\%\textgreater{}\%}
  \FunctionTok{summarise}\NormalTok{(}\AttributeTok{hourly\_avg\_intensity =} \FunctionTok{mean}\NormalTok{(average\_intensity))  }\SpecialCharTok{\%\textgreater{}\%}
  \FunctionTok{ggplot}\NormalTok{(}\FunctionTok{aes}\NormalTok{(}\AttributeTok{x=}\NormalTok{time, }\AttributeTok{y=}\NormalTok{hourly\_avg\_intensity, }\AttributeTok{fill=}\NormalTok{hourly\_avg\_intensity)) }\SpecialCharTok{+} 
  \FunctionTok{geom\_col}\NormalTok{() }\SpecialCharTok{+}
  \FunctionTok{labs}\NormalTok{(}\AttributeTok{title =} \StringTok{"Average intensity by hour"}\NormalTok{, }\AttributeTok{x=}\StringTok{"Hour"}\NormalTok{, }\AttributeTok{y=}\StringTok{"Average Intensity"}\NormalTok{) }\SpecialCharTok{+} 
  \FunctionTok{scale\_fill\_gradient}\NormalTok{(}\AttributeTok{low =} \StringTok{"\#fee8c8"}\NormalTok{, }\AttributeTok{high =} \StringTok{"\#d7301f"}\NormalTok{)}
\end{Highlighting}
\end{Shaded}

\includegraphics{Case-Study-Bellabeat_files/figure-latex/unnamed-chunk-2-2.pdf}

\begin{Shaded}
\begin{Highlighting}[]
\NormalTok{hours }\SpecialCharTok{\%\textgreater{}\%}
  \FunctionTok{group\_by}\NormalTok{(time) }\SpecialCharTok{\%\textgreater{}\%}
  \FunctionTok{summarise}\NormalTok{(}\AttributeTok{hourly\_calories =} \FunctionTok{mean}\NormalTok{(calories)) }\SpecialCharTok{\%\textgreater{}\%}
  \FunctionTok{ggplot}\NormalTok{(}\FunctionTok{aes}\NormalTok{(}\AttributeTok{x=}\NormalTok{time, }\AttributeTok{y=}\NormalTok{hourly\_calories, }\AttributeTok{fill=}\NormalTok{hourly\_calories)) }\SpecialCharTok{+} 
  \FunctionTok{geom\_col}\NormalTok{() }\SpecialCharTok{+}
  \FunctionTok{labs}\NormalTok{(}\AttributeTok{title =} \StringTok{"Hourly hourly\_calories"}\NormalTok{, }\AttributeTok{x=}\StringTok{"Hour"}\NormalTok{, }\AttributeTok{y=}\StringTok{"Calories"}\NormalTok{) }\SpecialCharTok{+} 
  \FunctionTok{scale\_fill\_gradient2}\NormalTok{(}\AttributeTok{low =} \StringTok{"\#f7fcfd"}\NormalTok{, }\AttributeTok{high =} \StringTok{"blue"}\NormalTok{)}
\end{Highlighting}
\end{Shaded}

\includegraphics{Case-Study-Bellabeat_files/figure-latex/unnamed-chunk-2-3.pdf}

\begin{Shaded}
\begin{Highlighting}[]
\NormalTok{hours }\SpecialCharTok{\%\textgreater{}\%}
  \FunctionTok{group\_by}\NormalTok{(time) }\SpecialCharTok{\%\textgreater{}\%}
  \FunctionTok{summarise}\NormalTok{(}\AttributeTok{hourly\_avg\_steps =} \FunctionTok{mean}\NormalTok{(step\_total)) }\SpecialCharTok{\%\textgreater{}\%}
  \FunctionTok{ggplot}\NormalTok{(}\FunctionTok{aes}\NormalTok{(}\AttributeTok{x=}\NormalTok{time, }\AttributeTok{y=}\NormalTok{hourly\_avg\_steps, }\AttributeTok{fill=}\NormalTok{hourly\_avg\_steps)) }\SpecialCharTok{+} 
  \FunctionTok{geom\_col}\NormalTok{() }\SpecialCharTok{+}
  \FunctionTok{labs}\NormalTok{(}\AttributeTok{title =} \StringTok{"Hourly step\_total"}\NormalTok{, }\AttributeTok{x=}\StringTok{"Hour"}\NormalTok{, }\AttributeTok{y=}\StringTok{"Average\_steps"}\NormalTok{) }\SpecialCharTok{+} 
  \FunctionTok{scale\_fill\_gradient2}\NormalTok{(}\AttributeTok{low =} \StringTok{"\#c7e9c0"}\NormalTok{, }\AttributeTok{high =} \StringTok{"\#238b45"}\NormalTok{)}
\end{Highlighting}
\end{Shaded}

\includegraphics{Case-Study-Bellabeat_files/figure-latex/unnamed-chunk-2-4.pdf}
We were able to observe a few hourly trends:

\emph{first peak between 12:00pm and 1:00pm }greatest peak around the
18th hour, or 6:00pm \emph{activity picks up at about 5:00am and rapidly
lowers at about 8:00pm }activity was lowest between 12:00am and 4:00am

We thought it would be interesting to graph each participant's maximum
hourly average intensity in order to see when people tended to exert
themselves the most.

\begin{Shaded}
\begin{Highlighting}[]
\NormalTok{hours }\SpecialCharTok{\%\textgreater{}\%}
  \FunctionTok{group\_by}\NormalTok{(id, time) }\SpecialCharTok{\%\textgreater{}\%}
  \FunctionTok{summarise}\NormalTok{(}\AttributeTok{hourly\_avg\_intensity =} \FunctionTok{mean}\NormalTok{(average\_intensity))  }\SpecialCharTok{\%\textgreater{}\%}
  \FunctionTok{group\_by}\NormalTok{(id) }\SpecialCharTok{\%\textgreater{}\%}
  \FunctionTok{mutate}\NormalTok{(}\AttributeTok{min\_intensity =} \FunctionTok{min}\NormalTok{(hourly\_avg\_intensity),}
            \AttributeTok{max\_intensity =} \FunctionTok{max}\NormalTok{(hourly\_avg\_intensity)) }\SpecialCharTok{\%\textgreater{}\%}
  \FunctionTok{filter}\NormalTok{(hourly\_avg\_intensity }\SpecialCharTok{==}\NormalTok{ max\_intensity) }\SpecialCharTok{\%\textgreater{}\%}
  \FunctionTok{ungroup}\NormalTok{() }\SpecialCharTok{\%\textgreater{}\%}
  \FunctionTok{ggplot}\NormalTok{(}\FunctionTok{aes}\NormalTok{(}\AttributeTok{x=}\NormalTok{time, }\AttributeTok{y=}\NormalTok{max\_intensity, }\AttributeTok{fill=}\NormalTok{max\_intensity)) }\SpecialCharTok{+} 
  \FunctionTok{geom\_col}\NormalTok{() }\SpecialCharTok{+} 
  \FunctionTok{labs}\NormalTok{(}\AttributeTok{title =} \StringTok{"Hourly max intensities"}\NormalTok{, }\AttributeTok{x=}\StringTok{"Time (hr)"}\NormalTok{, }\AttributeTok{y=}\StringTok{"Max intensities"}\NormalTok{) }\SpecialCharTok{+} 
  \FunctionTok{scale\_fill\_gradient}\NormalTok{(}\AttributeTok{low =} \StringTok{"\#fee8c8"}\NormalTok{, }\AttributeTok{high =} \StringTok{"\#d7301f"}\NormalTok{) }
\end{Highlighting}
\end{Shaded}

\begin{verbatim}
## `summarise()` has grouped output by 'id'. You can override using the `.groups`
## argument.
\end{verbatim}

\includegraphics{Case-Study-Bellabeat_files/figure-latex/unnamed-chunk-3-1.pdf}

Interestingly, when you graph the each participants' maximum intensity
in an hour out of all their measured hours, the trend has two peaks: the
first is around 8:00am and the second around 7:00pm. This suggests that
users exercise at a wide variety of times

\hypertarget{daily-stuff}{%
\paragraph{Daily stuff}\label{daily-stuff}}

Just as with the hourly merged dataset, we added a new column to keep
track of the day of the month to make our analysis easier.

\begin{Shaded}
\begin{Highlighting}[]
\NormalTok{daily\_activity }\OtherTok{\textless{}{-}}\NormalTok{ daily\_activity }\SpecialCharTok{\%\textgreater{}\%}
  \FunctionTok{mutate}\NormalTok{(}\AttributeTok{day =} \FunctionTok{day}\NormalTok{(activity\_date))}

\NormalTok{daily\_activity\_sleep }\OtherTok{\textless{}{-}}\NormalTok{ daily\_activity\_sleep }\SpecialCharTok{\%\textgreater{}\%}
  \FunctionTok{mutate}\NormalTok{(}\AttributeTok{day =} \FunctionTok{day}\NormalTok{(activity\_date))}
\end{Highlighting}
\end{Shaded}

Next, we found the correlations between the different measures.

We visualized this using a correlation heat map.

\begin{Shaded}
\begin{Highlighting}[]
\NormalTok{daily\_activity\_sleep }\SpecialCharTok{\%\textgreater{}\%}
  \FunctionTok{group\_by}\NormalTok{(id, activity\_date) }\SpecialCharTok{\%\textgreater{}\%}
  \FunctionTok{mutate}\NormalTok{(}\AttributeTok{total\_minutes=}\FunctionTok{mean}\NormalTok{(}\FunctionTok{sum}\NormalTok{(sedentary\_minutes,lightly\_active\_minutes,fairly\_active\_minutes,very\_active\_minutes))) }\SpecialCharTok{\%\textgreater{}\%}
  \FunctionTok{ungroup}\NormalTok{() }\SpecialCharTok{\%\textgreater{}\%}
  \FunctionTok{select}\NormalTok{(}\SpecialCharTok{{-}}\NormalTok{id, }\SpecialCharTok{{-}}\NormalTok{date, }\SpecialCharTok{{-}}\NormalTok{tracker\_distance, }\SpecialCharTok{{-}}\NormalTok{logged\_activities\_distance, }\SpecialCharTok{{-}}\NormalTok{activity\_date, }\SpecialCharTok{{-}}\NormalTok{time, }\SpecialCharTok{{-}}\NormalTok{total\_sleep\_records,}
         \SpecialCharTok{{-}}\NormalTok{sleep\_day, }\SpecialCharTok{{-}}\NormalTok{day)  }\SpecialCharTok{\%\textgreater{}\%}
  \FunctionTok{rename}\NormalTok{(}\AttributeTok{sedentary\_min=}\NormalTok{sedentary\_minutes, }\AttributeTok{light\_min=}\NormalTok{lightly\_active\_minutes,            }\CommentTok{\# renaming to reduce clutter}
         \AttributeTok{fairly\_min=}\NormalTok{fairly\_active\_minutes, }\AttributeTok{very\_min=}\NormalTok{very\_active\_minutes,}
         \AttributeTok{sedentary\_dist=}\NormalTok{sedentary\_active\_distance, }\AttributeTok{light\_dist=}\NormalTok{light\_active\_distance,}
         \AttributeTok{moderately\_dist=}\NormalTok{moderately\_active\_distance, }\AttributeTok{very\_dist=}\NormalTok{very\_active\_distance,}
         \AttributeTok{min\_asleep=}\NormalTok{total\_minutes\_asleep, }\AttributeTok{min\_in\_bed=}\NormalTok{total\_time\_in\_bed, }
         \AttributeTok{total\_min=}\NormalTok{total\_minutes, }\AttributeTok{total\_dist =}\NormalTok{ total\_distance) }\SpecialCharTok{\%\textgreater{}\%}
  \FunctionTok{ggcorr}\NormalTok{(}\AttributeTok{layout.exp =} \DecValTok{2}\NormalTok{, }\AttributeTok{label =}\NormalTok{ T, }\AttributeTok{label\_size =} \DecValTok{3}\NormalTok{, }\AttributeTok{label\_round =} \DecValTok{2}\NormalTok{, }\AttributeTok{label\_alpha =}\NormalTok{ T, }\AttributeTok{hjust =}\NormalTok{ .}\DecValTok{9}\NormalTok{, }\AttributeTok{size =} \FloatTok{4.5}\NormalTok{, }\AttributeTok{color =} \StringTok{"grey25"}\NormalTok{)}
\end{Highlighting}
\end{Shaded}

\includegraphics{Case-Study-Bellabeat_files/figure-latex/heat map of daily-1.pdf}

Observations:

\emph{calories burned had the strongest positive association with very
active minutes and total distance, suggesting that either being very
active for a longer period of time or traveling a greater distance each
day is related to higher calorie usage, although the moderate strength
of these correlations means it will not necessarily cause calorie
burning }although very active distance has the strongest relationship
with total distance, moderately and lightly active distance and minutes
also have a moderately strong relationship as well *sedentary minutes,
time in bed, and total time had a significant positive association,
suggesting that much of the time accounted for by the data was in
inactivity

Distances traveled by users on average

We wanted to see how the distance users traveled on average every day,
as well as what proportion of the distance was spent in sedentary,
light, moderate, or very active levels of intensity.

\begin{Shaded}
\begin{Highlighting}[]
\NormalTok{daily\_activity }\SpecialCharTok{\%\textgreater{}\%}
  \FunctionTok{group\_by}\NormalTok{(id) }\SpecialCharTok{\%\textgreater{}\%}
  \FunctionTok{summarise}\NormalTok{(}\AttributeTok{sedentary\_distance=}\FunctionTok{mean}\NormalTok{(sedentary\_active\_distance),}\AttributeTok{sedentary\_minutes=}\FunctionTok{mean}\NormalTok{(sedentary\_minutes),}
            \AttributeTok{lightly\_distance=}\FunctionTok{mean}\NormalTok{(light\_active\_distance),}\AttributeTok{lightly\_minutes=}\FunctionTok{mean}\NormalTok{(lightly\_active\_minutes),}
            \AttributeTok{moderately\_distance=}\FunctionTok{mean}\NormalTok{(moderately\_active\_distance),}\AttributeTok{moderate\_minutes=}\FunctionTok{mean}\NormalTok{(fairly\_active\_minutes),}
            \AttributeTok{very\_distance=}\FunctionTok{mean}\NormalTok{(very\_active\_distance),}\AttributeTok{very\_minutes=}\FunctionTok{mean}\NormalTok{(very\_active\_minutes),}
            \AttributeTok{total\_distance=}\FunctionTok{mean}\NormalTok{(total\_distance)) }\SpecialCharTok{\%\textgreater{}\%}
  \FunctionTok{pivot\_longer}\NormalTok{(}\AttributeTok{names\_to =} \StringTok{"mean\_distance"}\NormalTok{, }
               \AttributeTok{values\_to =} \StringTok{"distance"}\NormalTok{, }
               \FunctionTok{c}\NormalTok{(sedentary\_distance,lightly\_distance,moderately\_distance,very\_distance)) }\SpecialCharTok{\%\textgreater{}\%}
  \FunctionTok{mutate}\NormalTok{(}\AttributeTok{mean\_distance=}\FunctionTok{factor}\NormalTok{(mean\_distance, }
                              \AttributeTok{levels =} \FunctionTok{c}\NormalTok{(}\StringTok{"sedentary\_distance"}\NormalTok{, }\StringTok{"lightly\_distance"}\NormalTok{, }\StringTok{"moderately\_distance"}\NormalTok{,}\StringTok{"very\_distance"}\NormalTok{),}
                              \AttributeTok{ordered =} \ConstantTok{TRUE}\NormalTok{)) }\SpecialCharTok{\%\textgreater{}\%}
  \FunctionTok{ungroup}\NormalTok{() }\SpecialCharTok{\%\textgreater{}\%}
  \FunctionTok{arrange}\NormalTok{(}\FunctionTok{desc}\NormalTok{(total\_distance)) }\SpecialCharTok{\%\textgreater{}\%}
  \FunctionTok{mutate}\NormalTok{(}\AttributeTok{id =} \FunctionTok{factor}\NormalTok{(id, }\AttributeTok{levels =} \FunctionTok{unique}\NormalTok{(id))) }\SpecialCharTok{\%\textgreater{}\%}
  \FunctionTok{ggplot}\NormalTok{() }\SpecialCharTok{+}
  \FunctionTok{geom\_col}\NormalTok{(}\AttributeTok{mapping =} \FunctionTok{aes}\NormalTok{(}\AttributeTok{x=}\NormalTok{id,}\AttributeTok{y=}\NormalTok{distance,}\AttributeTok{fill=}\NormalTok{mean\_distance)) }\SpecialCharTok{+}
  \FunctionTok{coord\_flip}\NormalTok{() }\SpecialCharTok{+} 
  \FunctionTok{labs}\NormalTok{(}\AttributeTok{title =} \StringTok{"Participants by active distance"}\NormalTok{) }\SpecialCharTok{+}
  \FunctionTok{theme}\NormalTok{(}\AttributeTok{axis.text.y=}\FunctionTok{element\_blank}\NormalTok{())}
\end{Highlighting}
\end{Shaded}

\includegraphics{Case-Study-Bellabeat_files/figure-latex/unnamed-chunk-4-1.pdf}

\begin{Shaded}
\begin{Highlighting}[]
\DocumentationTok{\#\# Proportion}
\NormalTok{daily\_activity }\SpecialCharTok{\%\textgreater{}\%}
  \FunctionTok{group\_by}\NormalTok{(id) }\SpecialCharTok{\%\textgreater{}\%}
  \FunctionTok{mutate}\NormalTok{(}\AttributeTok{prop\_sedentary =}\NormalTok{ sedentary\_active\_distance }\SpecialCharTok{/} \FunctionTok{sum}\NormalTok{(sedentary\_active\_distance,light\_active\_distance,}
\NormalTok{                    moderately\_active\_distance,very\_active\_distance),}
         \AttributeTok{prop\_light =}\NormalTok{ light\_active\_distance }\SpecialCharTok{/} \FunctionTok{sum}\NormalTok{(sedentary\_active\_distance,light\_active\_distance,}
\NormalTok{                    moderately\_active\_distance,very\_active\_distance),}
         \AttributeTok{prop\_moderately =}\NormalTok{ moderately\_active\_distance }\SpecialCharTok{/} \FunctionTok{sum}\NormalTok{(sedentary\_active\_distance,light\_active\_distance,}
\NormalTok{                    moderately\_active\_distance,very\_active\_distance),}
         \AttributeTok{prop\_very =}\NormalTok{ very\_active\_distance }\SpecialCharTok{/} \FunctionTok{sum}\NormalTok{(sedentary\_active\_distance,light\_active\_distance,}
\NormalTok{                    moderately\_active\_distance,very\_active\_distance),}
            \AttributeTok{total\_distance=}\FunctionTok{mean}\NormalTok{(total\_distance)) }\SpecialCharTok{\%\textgreater{}\%}
  \FunctionTok{pivot\_longer}\NormalTok{(}\AttributeTok{names\_to =} \StringTok{"prop\_distance"}\NormalTok{, }\AttributeTok{values\_to =} \StringTok{"proportion"}\NormalTok{, prop\_sedentary}\SpecialCharTok{:}\NormalTok{prop\_very) }\SpecialCharTok{\%\textgreater{}\%}
  \FunctionTok{mutate}\NormalTok{(}\AttributeTok{prop\_distance=}\FunctionTok{factor}\NormalTok{(prop\_distance, }
                              \AttributeTok{levels =} \FunctionTok{c}\NormalTok{(}\StringTok{"prop\_sedentary"}\NormalTok{, }\StringTok{"prop\_light"}\NormalTok{, }\StringTok{"prop\_moderately"}\NormalTok{,}\StringTok{"prop\_very"}\NormalTok{),}
                              \AttributeTok{ordered =} \ConstantTok{TRUE}\NormalTok{)) }\SpecialCharTok{\%\textgreater{}\%}
  \FunctionTok{ungroup}\NormalTok{() }\SpecialCharTok{\%\textgreater{}\%}
  \FunctionTok{arrange}\NormalTok{(}\FunctionTok{desc}\NormalTok{(total\_distance)) }\SpecialCharTok{\%\textgreater{}\%}
  \FunctionTok{mutate}\NormalTok{(}\AttributeTok{id =} \FunctionTok{factor}\NormalTok{(id, }\AttributeTok{levels=}\FunctionTok{unique}\NormalTok{(id))) }\SpecialCharTok{\%\textgreater{}\%}
  \FunctionTok{ggplot}\NormalTok{() }\SpecialCharTok{+}
  \FunctionTok{geom\_col}\NormalTok{(}\AttributeTok{mapping =} \FunctionTok{aes}\NormalTok{(}\AttributeTok{x=}\NormalTok{id,}\AttributeTok{y=}\NormalTok{proportion,}\AttributeTok{fill=}\NormalTok{prop\_distance)) }\SpecialCharTok{+}
  \FunctionTok{coord\_flip}\NormalTok{() }\SpecialCharTok{+} 
  \FunctionTok{labs}\NormalTok{(}\AttributeTok{title =} \StringTok{"Participants by proportion of active distance"}\NormalTok{) }\SpecialCharTok{+}
  \FunctionTok{theme}\NormalTok{(}\AttributeTok{axis.text.y=}\FunctionTok{element\_blank}\NormalTok{())}
\end{Highlighting}
\end{Shaded}

\includegraphics{Case-Study-Bellabeat_files/figure-latex/unnamed-chunk-4-2.pdf}

Observations:

\emph{users are com from a wide variety of different levels of activity
}most distance is covered by light distance (e.g., walking)) (maybe
suggesting that the vast majority of people use fitbit for light
activity, with very active activity not far behind. This goes hand in
hand with the finding of calories and getting to work; if someone puts
in even a little effort, they will see change. And we see the majority
of users following this. Maybe advertising how easy it is to get in
shape, rewarding those who do light activity (e.g., in games, in
notifications that ``you've walked this far today!'')

\begin{Shaded}
\begin{Highlighting}[]
\CommentTok{\# First, we created a data frame that includes the day of the week}
\NormalTok{days }\OtherTok{\textless{}{-}}\NormalTok{ daily\_activity\_sleep }\SpecialCharTok{\%\textgreater{}\%}
  \FunctionTok{mutate}\NormalTok{(}\AttributeTok{day\_of\_week =} \FunctionTok{wday}\NormalTok{(activity\_date),}
         \AttributeTok{day\_of\_week =} \FunctionTok{factor}\NormalTok{(}\AttributeTok{x=}\NormalTok{day\_of\_week, }
                              \AttributeTok{levels =} \FunctionTok{c}\NormalTok{(}\DecValTok{1}\NormalTok{,}\DecValTok{2}\NormalTok{,}\DecValTok{3}\NormalTok{,}\DecValTok{4}\NormalTok{,}\DecValTok{5}\NormalTok{,}\DecValTok{6}\NormalTok{,}\DecValTok{7}\NormalTok{), }
                              \AttributeTok{labels =} \FunctionTok{c}\NormalTok{(}\StringTok{"Sunday"}\NormalTok{,}\StringTok{"Monday"}\NormalTok{,}\StringTok{"Tuesday"}\NormalTok{,}\StringTok{"Wednesday"}\NormalTok{,}\StringTok{"Thursday"}\NormalTok{,}\StringTok{"Friday"}\NormalTok{,}\StringTok{"Saturday"}\NormalTok{),}
                              \AttributeTok{ordered =}\NormalTok{ T))}
\FunctionTok{ggarrange}\NormalTok{(}
  \FunctionTok{ggplot}\NormalTok{(}\AttributeTok{data=}\NormalTok{days) }\SpecialCharTok{+}
    \FunctionTok{geom\_violin}\NormalTok{(}\AttributeTok{scale =} \StringTok{"area"}\NormalTok{, }\AttributeTok{mapping =} \FunctionTok{aes}\NormalTok{(}\AttributeTok{x=}\NormalTok{day\_of\_week,}\AttributeTok{y=}\NormalTok{total\_minutes\_asleep), }\AttributeTok{fill =} \StringTok{"pink"}\NormalTok{) }\SpecialCharTok{+}
    \FunctionTok{stat\_summary}\NormalTok{(}\AttributeTok{mapping =} \FunctionTok{aes}\NormalTok{(}\AttributeTok{x=}\NormalTok{day\_of\_week,}\AttributeTok{y=}\NormalTok{total\_minutes\_asleep), }\AttributeTok{fun =}\NormalTok{ median, }\AttributeTok{geom =} \StringTok{\textquotesingle{}point\textquotesingle{}}\NormalTok{, }\AttributeTok{color =} \StringTok{"red"}\NormalTok{) }\SpecialCharTok{+}
    \FunctionTok{stat\_mean}\NormalTok{(}\AttributeTok{mapping =} \FunctionTok{aes}\NormalTok{(}\AttributeTok{x=}\NormalTok{day\_of\_week,}\AttributeTok{y=}\NormalTok{total\_minutes\_asleep)) }\SpecialCharTok{+}
    \FunctionTok{geom\_hline}\NormalTok{(}\FunctionTok{aes}\NormalTok{(}\AttributeTok{yintercept=}\NormalTok{(}\DecValTok{8}\SpecialCharTok{*}\DecValTok{60}\NormalTok{))) }\SpecialCharTok{+}
    \FunctionTok{labs}\NormalTok{(}\AttributeTok{title =} \StringTok{"Minutes asleep by day of the week"}\NormalTok{, }\AttributeTok{x=} \StringTok{""}\NormalTok{, }\AttributeTok{y =} \StringTok{""}\NormalTok{) }\SpecialCharTok{+}
    \FunctionTok{annotate}\NormalTok{(}\StringTok{"text"}\NormalTok{, }\AttributeTok{x=}\DecValTok{2}\NormalTok{,}\AttributeTok{y=}\FunctionTok{c}\NormalTok{(}\DecValTok{585}\NormalTok{,}\DecValTok{615}\NormalTok{), }\AttributeTok{label =} \FunctionTok{c}\NormalTok{(}\StringTok{"mean"}\NormalTok{, }\StringTok{"median"}\NormalTok{), }\AttributeTok{color =} \FunctionTok{c}\NormalTok{(}\StringTok{"black"}\NormalTok{,}\StringTok{"red"}\NormalTok{)) }\SpecialCharTok{+}
    \FunctionTok{theme}\NormalTok{(}\AttributeTok{axis.text.x =} \FunctionTok{element\_text}\NormalTok{(}\AttributeTok{angle =} \DecValTok{45}\NormalTok{, }\AttributeTok{vjust =}\NormalTok{ .}\DecValTok{5}\NormalTok{, }\AttributeTok{hjust =} \DecValTok{1}\NormalTok{)),}
  \FunctionTok{ggplot}\NormalTok{(}\AttributeTok{data =}\NormalTok{ days) }\SpecialCharTok{+}
    \FunctionTok{geom\_violin}\NormalTok{(}\AttributeTok{mapping =} \FunctionTok{aes}\NormalTok{(}\AttributeTok{x=}\NormalTok{day\_of\_week,}\AttributeTok{y=}\NormalTok{total\_steps), }\AttributeTok{fill =} \StringTok{"\#238b45"}\NormalTok{,)  }\SpecialCharTok{+}
    \FunctionTok{stat\_summary}\NormalTok{(}\AttributeTok{mapping =} \FunctionTok{aes}\NormalTok{(}\AttributeTok{x=}\NormalTok{day\_of\_week,}\AttributeTok{y=}\NormalTok{total\_steps), }\AttributeTok{fun =}\NormalTok{ median, }\AttributeTok{geom =} \StringTok{\textquotesingle{}point\textquotesingle{}}\NormalTok{, }\AttributeTok{color =} \StringTok{"red"}\NormalTok{) }\SpecialCharTok{+}
    \FunctionTok{stat\_mean}\NormalTok{(}\AttributeTok{mapping =} \FunctionTok{aes}\NormalTok{(}\AttributeTok{x=}\NormalTok{day\_of\_week,}\AttributeTok{y=}\NormalTok{total\_steps)) }\SpecialCharTok{+}
    \FunctionTok{geom\_hline}\NormalTok{(}\FunctionTok{aes}\NormalTok{(}\AttributeTok{yintercept=}\NormalTok{(}\DecValTok{7500}\NormalTok{))) }\SpecialCharTok{+}
    \FunctionTok{labs}\NormalTok{(}\AttributeTok{title =} \StringTok{"Total steps by day of the week"}\NormalTok{, }\AttributeTok{x=} \StringTok{""}\NormalTok{, }\AttributeTok{y =} \StringTok{""}\NormalTok{) }\SpecialCharTok{+}
    \FunctionTok{theme}\NormalTok{(}\AttributeTok{axis.text.x =} \FunctionTok{element\_text}\NormalTok{(}\AttributeTok{angle =} \DecValTok{45}\NormalTok{, }\AttributeTok{vjust =}\NormalTok{ .}\DecValTok{5}\NormalTok{, }\AttributeTok{hjust =} \DecValTok{1}\NormalTok{))}
\NormalTok{    )}
\end{Highlighting}
\end{Shaded}

\includegraphics{Case-Study-Bellabeat_files/figure-latex/unnamed-chunk-5-1.pdf}
On average:

\emph{although users sleep the most on Sundays, they do not reach 8
hours of sleep on any day of the week }users approach 7500 to 10,000
steps per day of the week

Minutes per week

According to the CDC's recommendations on
\href{https://www.cdc.gov/physicalactivity/basics/age-chart.html}{Physical
Activity for Different Groups}, an adult should have ``At least 150
minutes a week of moderate intensity activity such as brisk walking.''
we took to find if the our Fitbit data saw users actually hitting those
goals, or if it could be something to target.

\begin{Shaded}
\begin{Highlighting}[]
\NormalTok{weekly\_meet\_count }\OtherTok{\textless{}{-}}\NormalTok{ days }\SpecialCharTok{\%\textgreater{}\%}
  \FunctionTok{group\_by}\NormalTok{(id) }\SpecialCharTok{\%\textgreater{}\%}
  \FunctionTok{mutate}\NormalTok{(}\AttributeTok{week =} \FunctionTok{week}\NormalTok{(activity\_date)) }\SpecialCharTok{\%\textgreater{}\%}
  \FunctionTok{group\_by}\NormalTok{(id, week) }\SpecialCharTok{\%\textgreater{}\%}
  \FunctionTok{mutate}\NormalTok{(}\AttributeTok{full\_week =} \FunctionTok{n}\NormalTok{() }\SpecialCharTok{\textgreater{}=} \DecValTok{7}\NormalTok{) }\SpecialCharTok{\%\textgreater{}\%}          \CommentTok{\# We did this only to include only complete weeks that had all 7 days}
  \FunctionTok{filter}\NormalTok{(full\_week }\SpecialCharTok{==} \ConstantTok{TRUE}\NormalTok{) }\SpecialCharTok{\%\textgreater{}\%}
  \FunctionTok{group\_by}\NormalTok{(id, week) }\SpecialCharTok{\%\textgreater{}\%}
  \FunctionTok{mutate}\NormalTok{(}\AttributeTok{active\_min =} \FunctionTok{sum}\NormalTok{(fairly\_active\_minutes, very\_active\_minutes),}
            \AttributeTok{meets =}\NormalTok{ active\_min }\SpecialCharTok{\textgreater{}} \DecValTok{150}\NormalTok{) }\SpecialCharTok{\%\textgreater{}\%}
  \FunctionTok{ggplot}\NormalTok{() }\SpecialCharTok{+} \FunctionTok{geom\_bar}\NormalTok{(}\AttributeTok{mapping =} \FunctionTok{aes}\NormalTok{(}\AttributeTok{x=}\NormalTok{meets,}\AttributeTok{fill=}\NormalTok{meets)) }\SpecialCharTok{+}
  \FunctionTok{labs}\NormalTok{(}\AttributeTok{title =} \StringTok{"Number of users"}\NormalTok{, }\AttributeTok{x =} \StringTok{""}\NormalTok{, }\AttributeTok{y =} \StringTok{""}\NormalTok{)}

\NormalTok{weekly\_meet }\OtherTok{\textless{}{-}}\NormalTok{ days }\SpecialCharTok{\%\textgreater{}\%}
  \FunctionTok{group\_by}\NormalTok{(id) }\SpecialCharTok{\%\textgreater{}\%}
  \FunctionTok{mutate}\NormalTok{(}\AttributeTok{week =} \FunctionTok{week}\NormalTok{(activity\_date)) }\SpecialCharTok{\%\textgreater{}\%}
  \FunctionTok{group\_by}\NormalTok{(id, week) }\SpecialCharTok{\%\textgreater{}\%}
  \FunctionTok{mutate}\NormalTok{(}\AttributeTok{full\_week =} \FunctionTok{n}\NormalTok{() }\SpecialCharTok{\textgreater{}=} \DecValTok{7}\NormalTok{) }\SpecialCharTok{\%\textgreater{}\%}
  \FunctionTok{filter}\NormalTok{(full\_week }\SpecialCharTok{==} \ConstantTok{TRUE}\NormalTok{) }\SpecialCharTok{\%\textgreater{}\%}
  \FunctionTok{group\_by}\NormalTok{(id, week) }\SpecialCharTok{\%\textgreater{}\%}
  \FunctionTok{mutate}\NormalTok{(}\AttributeTok{active\_min =} \FunctionTok{sum}\NormalTok{(fairly\_active\_minutes, very\_active\_minutes),}
            \AttributeTok{meets =}\NormalTok{ active\_min }\SpecialCharTok{\textgreater{}} \DecValTok{150}\NormalTok{) }\SpecialCharTok{\%\textgreater{}\%}
  \FunctionTok{ggplot}\NormalTok{() }\SpecialCharTok{+} 
  \FunctionTok{geom\_bar}\NormalTok{(}\AttributeTok{mapping =} \FunctionTok{aes}\NormalTok{(}\AttributeTok{x=}\NormalTok{week,}\AttributeTok{fill=}\NormalTok{meets), }\AttributeTok{position =} \StringTok{"dodge"}\NormalTok{) }\SpecialCharTok{+}
  \FunctionTok{labs}\NormalTok{(}\AttributeTok{title =} \StringTok{"Number of users per week"}\NormalTok{, }\AttributeTok{x =} \StringTok{""}\NormalTok{, }\AttributeTok{y =} \StringTok{""}\NormalTok{)}



\NormalTok{weekly\_meet\_arrange }\OtherTok{\textless{}{-}} \FunctionTok{ggarrange}\NormalTok{(}\AttributeTok{common.legend =} \ConstantTok{TRUE}\NormalTok{, }\AttributeTok{widths =} \FunctionTok{c}\NormalTok{(}\DecValTok{1}\NormalTok{,}\DecValTok{2}\NormalTok{), }\AttributeTok{legend =} \StringTok{"bottom"}\NormalTok{, weekly\_meet\_count, weekly\_meet)}
\FunctionTok{annotate\_figure}\NormalTok{(weekly\_meet\_arrange, }\AttributeTok{top =} \FunctionTok{text\_grob}\NormalTok{(}\StringTok{"Do you meet the CDC recommendations of weekly activity?"}\NormalTok{, }
               \AttributeTok{color =} \StringTok{"red"}\NormalTok{, }\AttributeTok{face =} \StringTok{"bold"}\NormalTok{, }\AttributeTok{size =} \DecValTok{14}\NormalTok{))}
\end{Highlighting}
\end{Shaded}

\includegraphics{Case-Study-Bellabeat_files/figure-latex/minutes per week-1.pdf}

\begin{Shaded}
\begin{Highlighting}[]
\DocumentationTok{\#\# After removing incomplete weeks, most users seem to reach the appropriate levels of active minutes. Let\textquotesingle{}s see on a daily basis}

\DocumentationTok{\#\#\# 50\% of people }

\NormalTok{daily\_meet\_count }\OtherTok{\textless{}{-}}\NormalTok{ days }\SpecialCharTok{\%\textgreater{}\%}
  \FunctionTok{group\_by}\NormalTok{(id, activity\_date) }\SpecialCharTok{\%\textgreater{}\%}
  \FunctionTok{mutate}\NormalTok{(}\AttributeTok{active\_min =} \FunctionTok{sum}\NormalTok{(fairly\_active\_minutes, very\_active\_minutes),}
         \AttributeTok{meets =}\NormalTok{ active\_min }\SpecialCharTok{\textgreater{}}\NormalTok{ (}\DecValTok{150}\SpecialCharTok{/}\DecValTok{7}\NormalTok{)) }\SpecialCharTok{\%\textgreater{}\%}
  \FunctionTok{ggplot}\NormalTok{() }\SpecialCharTok{+} \FunctionTok{geom\_bar}\NormalTok{(}\AttributeTok{mapping =} \FunctionTok{aes}\NormalTok{(}\AttributeTok{x=}\NormalTok{meets,}\AttributeTok{fill=}\NormalTok{meets)) }\SpecialCharTok{+}
  \FunctionTok{labs}\NormalTok{(}\AttributeTok{title =} \StringTok{"Number of users"}\NormalTok{, }\AttributeTok{x =} \StringTok{""}\NormalTok{, }\AttributeTok{y =} \StringTok{""}\NormalTok{)}

\NormalTok{daily\_meet }\OtherTok{\textless{}{-}}\NormalTok{ days }\SpecialCharTok{\%\textgreater{}\%}
  \FunctionTok{group\_by}\NormalTok{(id, activity\_date) }\SpecialCharTok{\%\textgreater{}\%}
  \FunctionTok{mutate}\NormalTok{(}\AttributeTok{active\_min =} \FunctionTok{sum}\NormalTok{(fairly\_active\_minutes, very\_active\_minutes),}
         \AttributeTok{meets =}\NormalTok{ active\_min }\SpecialCharTok{\textgreater{}}\NormalTok{ (}\DecValTok{150}\SpecialCharTok{/}\DecValTok{7}\NormalTok{)) }\SpecialCharTok{\%\textgreater{}\%}
  \FunctionTok{ggplot}\NormalTok{() }\SpecialCharTok{+} \FunctionTok{geom\_bar}\NormalTok{(}\AttributeTok{mapping =} \FunctionTok{aes}\NormalTok{(}\AttributeTok{x=}\NormalTok{day\_of\_week,}\AttributeTok{fill=}\NormalTok{meets), }\AttributeTok{position =} \StringTok{"dodge"}\NormalTok{) }\SpecialCharTok{+}
  \FunctionTok{labs}\NormalTok{(}\AttributeTok{title =} \StringTok{"Number of users per day of the week"}\NormalTok{, }
       \AttributeTok{x =} \StringTok{""}\NormalTok{, }\AttributeTok{y =} \StringTok{""}\NormalTok{)}

\NormalTok{weekly\_meet\_arrange }\OtherTok{\textless{}{-}} \FunctionTok{ggarrange}\NormalTok{(}\AttributeTok{common.legend =} \ConstantTok{TRUE}\NormalTok{, }\AttributeTok{widths =} \FunctionTok{c}\NormalTok{(}\DecValTok{1}\NormalTok{,}\DecValTok{2}\NormalTok{), }\AttributeTok{legend =} \StringTok{"bottom"}\NormalTok{, daily\_meet\_count, daily\_meet)}
\FunctionTok{annotate\_figure}\NormalTok{(weekly\_meet\_arrange, }\AttributeTok{top =} \FunctionTok{text\_grob}\NormalTok{(}\StringTok{"Do you meet the CDC recommendation of daily activity?"}\NormalTok{, }
               \AttributeTok{color =} \StringTok{"red"}\NormalTok{, }\AttributeTok{face =} \StringTok{"bold"}\NormalTok{, }\AttributeTok{size =} \DecValTok{14}\NormalTok{))}
\end{Highlighting}
\end{Shaded}

\includegraphics{Case-Study-Bellabeat_files/figure-latex/minutes per week-2.pdf}

Observations:

\emph{most users met the CDC weekly recommendations of weekly activity,
although that number waned as the weeks progressed }based on the daily
activity graphics, which had more data points to work with, there were
much more days that did not meet CDC recommendations. This might be due
to people having rest days, which people seemed to prefer to do on
Sundays and be the most active on Saturdays

(maybe encouraging a rest day for users for recomposition)

\hypertarget{act}{%
\subsection{Act}\label{act}}

\#\#\#\#Conclusions We can make several observations about these
correlational graphs:

\emph{There is a strong significant positive correlation between average
intensity, step total, and calories burned, as expected; the higher any
one of these variables are, the higher any of the other ones are as
well. }The average intensity by calories trend line starts off very
steep. This seems to suggest that relatively low activity intensity
might still have large benefits in terms of burning calories. This could
be an important point to touch on for users as it may persuade more
sedentary users to engage in light, or even moderate, levels of activity
intensity. This means that even walking more than one usually would is
very good.

We were able to observe a few hourly trends:

\emph{first peak between 12:00pm and 1:00pm }greatest peak around the
18th hour, or 6:00pm \emph{activity picks up at about 5:00am and rapidly
lowers at about 8:00pm }activity was lowest between 12:00am and 4:00am

Interestingly, when you graph the each participants' maximum intensity
in an hour out of all their measured hours, the trend has two peaks: the
first is around 8:00am and the second around 7:00pm. This suggests that
users exercise at a wide variety of times

Observations:

\emph{calories burned had the strongest positive association with very
active minutes and total distance, suggesting that either being very
active for a longer period of time or traveling a greater distance each
day is related to higher calorie usage, although the moderate strength
of these correlations means it will not necessarily cause calorie
burning }although very active distance has the strongest relationship
with total distance, moderately and lightly active distance and minutes
also have a moderately strong relationship as well *sedentary minutes,
time in bed, and total time had a significant positive association,
suggesting that much of the time accounted for by the data was in
inactivity

Observations:

\emph{users are com from a wide variety of different levels of activity
}most distance is covered by light distance (e.g., walking)) (maybe
suggesting that the vast majority of people use fitbit for light
activity, with very active activity not far behind. This goes hand in
hand with the finding of calories and getting to work; if someone puts
in even a little effort, they will see change. And we see the majority
of users following this. Maybe advertising how easy it is to get in
shape, rewarding those who do light activity (e.g., in games, in
notifications that ``you've walked this far today!'')

\#\#\#\#Application

\#\#\#\#Next steps

\#\#\#\#Additional data to use

Guiding questions ● What is your final conclusion based on your
analysis? ● How could your team and business apply your insights? ● What
next steps would you or your stakeholders take based on your findings? ●
Is there additional data you could use to expand on your findings?

\end{document}
